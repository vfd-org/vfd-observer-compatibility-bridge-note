% ======================================================================
%  Observer Compatibility as a Bridge from Geometric Invariants
%  to Measurement: Phase-Compatible Sampling of a Closure Residual
%
%  Author : Lee Smart
%  Affiliation : Vibrational Field Dynamics Institute
%  Date : December 4, 2025
%  License : MIT
%
%  Companion to: emergent-em-boundary-geometry
%  Build: latexmk -pdf observer-compatibility-bridge-note.tex
% ======================================================================

\documentclass[11pt,a4paper]{article}

% ── Packages ──────────────────────────────────────────────────────────
\usepackage[utf8]{inputenc}
\usepackage[T1]{fontenc}
\usepackage{lmodern}
\usepackage{amsmath,amssymb,amsthm}
\usepackage{geometry}
\usepackage{hyperref}
\usepackage{cleveref}
\usepackage{enumitem}
\usepackage{booktabs}
\usepackage{microtype}
\usepackage{xcolor}
\usepackage{tcolorbox}
\usepackage{graphicx}

% ── Page geometry ─────────────────────────────────────────────────────
\geometry{margin=1in}

% ── Hyperref setup ────────────────────────────────────────────────────
\hypersetup{
  colorlinks  = true,
  linkcolor   = blue!60!black,
  citecolor   = green!50!black,
  urlcolor    = blue!70!black,
  pdftitle    = {Observer Compatibility as a Bridge from Geometric
                 Invariants to Measurement},
  pdfauthor   = {Lee Smart},
}

% ── Theorem-like environments ─────────────────────────────────────────
\newtheorem{definition}{Definition}[section]
\newtheorem{principle}{Principle}[section]
\newtheorem{conjecture}{Conjecture}[section]
\newtheorem{remark}{Remark}[section]

% ── Convenience macros ────────────────────────────────────────────────
\newcommand{\Su}{\mathrm{SU}(2)}
\newcommand{\Sthree}{S^{3}}
\newcommand{\alphaFS}{\alpha}
\newcommand{\alphainv}{\alpha^{-1}}
\newcommand{\Hol}{\mathrm{Hol}}
\newcommand{\R}{\mathbb{R}}
\newcommand{\Z}{\mathbb{Z}}

% ── Title block ───────────────────────────────────────────────────────
\title{%
  \textbf{Observer Compatibility as a Bridge from\\
  Geometric Invariants to Measurement}\\[6pt]
  \large Phase-Compatible Sampling of a Closure Residual%
}

\author{%
  Lee Smart\\[2pt]
  \small Vibrational Field Dynamics Institute\\
  \small \href{mailto:contact@vibrationalfielddynamics.org}%
  {contact@vibrationalfielddynamics.org}\\
  \small \href{https://x.com/vfd_org}{@vfd\_org}%
}

\date{December 4, 2025}

% ======================================================================
\begin{document}
\maketitle

% ── Abstract ──────────────────────────────────────────────────────────
\begin{abstract}
The companion paper \cite{Smart2025geometry} derives a dimensionless
closure residual~$\alphaFS$ from the geometry of parallel transport on
$\Sthree \cong \Su$.  That residual is a timeless, observer-independent
invariant of the manifold; it requires no measurement apparatus for its
existence.  \emph{This} bridge note addresses a different question:
under what conditions can an embodied physical system---an
``observer''---produce a measurement that \emph{resolves} $\alphaFS$
coherently?

We introduce the notion of \textbf{observer compatibility}: a set of
phase-comparison constraints that a finite, cyclic measurement process
must satisfy to sample a geometric invariant without distorting it.
Compatibility is a property of the measurement channel, not of the
invariant itself; $\alphaFS$ is not observer-dependent.

As a concrete (and explicitly conjectural) application, we note that
biological neural systems exhibit a well-documented oscillatory band in
the 80--100\,Hz range.  We hypothesize that $\approx 87$\,Hz
represents a candidate \emph{compatibility point}---the lowest
frequency at which a biologically plausible phase-locked loop can
stably sample the closure residual over multiple cycles.  This
hypothesis is falsifiable and carries no medical, therapeutic, or
consciousness-theoretic claims.
\end{abstract}

% ── Scope disclaimer box ─────────────────────────────────────────────
\begin{tcolorbox}[
  colback   = gray!5,
  colframe  = gray!60,
  title     = {Scope of This Note},
  fonttitle = \bfseries\small,
  boxrule   = 0.4pt,
  arc       = 2pt
]
\small
\begin{itemize}[nosep,leftmargin=1.2em]
  \item This note does \textbf{not} derive $\alphaFS$ from biology.
  \item This note does \textbf{not} claim that consciousness creates
        physical constants.
  \item This note proposes a \emph{compatibility condition} for
        measurement---i.e.\ coherent sampling---of a pre-existing
        geometric invariant.
  \item Any mention of 87\,Hz is a \emph{hypothesis about sampling
        stability}, not a claim of universality.
\end{itemize}
\end{tcolorbox}

\bigskip

% ======================================================================
\section{Introduction}
\label{sec:intro}
% ======================================================================

A geometric theorem and a laboratory measurement live in different
categories.  The former is a statement about structure; the latter is a
physical process embedded in time, energy, and apparatus.  The companion
paper~\cite{Smart2025geometry} works entirely within the first category:
it shows that parallel transport around a distinguished closed path
$\gamma$ on $\Sthree$ produces a holonomy with a small, path-stable
\emph{angular} deficit.  Let $\Hol(\gamma)\in\Su$ denote the transport
holonomy and let $\theta(\gamma)\in[0,2\pi]$ be its associated rotation
angle (e.g.\ via the logarithm map on $\Su$).  The closure residual can
then be represented schematically as a dimensionless fraction
\begin{equation}
  \label{eq:alpha-def}
  \alphaFS \;\sim\; \frac{\theta(\gamma)}{2\pi}\,,
\end{equation}
with the companion paper specifying the distinguished class of paths and
normalisation conventions used to identify this fraction with the
fine-structure constant.  The derivation is purely geometric: no coupling
constants, no renormalization, no observers.

Yet the fine-structure constant \emph{is} measured---with extraordinary
precision~\cite{Tiesinga2021, Aoyama2020}---by physical systems that
occupy finite spatial extent, operate over finite time intervals, and
couple to electromagnetic fields at specific energies.  The question
this note addresses is:

\begin{quote}
\itshape
What structural conditions must a physical measurement process
satisfy in order to resolve a timeless geometric invariant without
making that invariant depend on the process itself?
\end{quote}

We call such conditions \textbf{observer compatibility}.  (Here and
throughout, ``observer'' denotes any physical measurement or sampling
process; no reference to consciousness or subjective experience is
intended.)  The purpose of this bridge note is to define observer
compatibility precisely, distinguish it sharply from observer
dependence, and sketch one falsifiable application to biological phase
coherence.

\paragraph{Companion paper.}
This note is a companion to~\cite{Smart2025geometry} and does not
repeat the geometric derivation.  Readers unfamiliar with the closure
residual construction should consult that paper first.

% ======================================================================
\section{Definitions}
\label{sec:defs}
% ======================================================================

We collect the key terms used throughout this note.  Each definition
is informal but intended to be precise enough for the arguments that
follow.

\begin{definition}[Closure invariant]
\label{def:closure-invariant}
A \emph{closure invariant} is a dimensionless scalar that arises as
the fractional holonomy deficit when a frame is parallel-transported
around a closed, non-contractible path on a compact manifold.  In the
companion paper, the manifold is $\Sthree \cong \Su$, the path~$\gamma$
is a great-circle geodesic in a distinguished homotopy class, and the
resulting closure invariant is identified with the fine-structure
constant~$\alphaFS$.
\end{definition}

\begin{definition}[Observer compatibility]
\label{def:observer-compat}
A measurement process is \emph{observer-compatible} with a closure
invariant~$c$ if the process satisfies a set of phase-comparison
constraints such that:
\begin{enumerate}[nosep,label=(\roman*)]
  \item the process completes at least one full cycle of the transport
        path that defines~$c$;
  \item the phase accumulated by the process over one cycle can be
        compared---without ambiguity modulo~$2\pi$---to the holonomy
        angle of~$\gamma$; and
  \item the sampling rate of the process is sufficient to distinguish
        $c$ from~$0$ given the noise floor of the physical channel.
\end{enumerate}
\end{definition}

\begin{definition}[Phase transport (discrete)]
\label{def:phase-transport}
Let $\{p_0, p_1, \ldots, p_{N-1}, p_N = p_0\}$ be an ordered set of
points on~$\Sthree$ approximating a closed geodesic~$\gamma$.  A
\emph{discrete phase transport rule}~$T$ assigns to each consecutive
pair $(p_k, p_{k+1})$ a rotation $R_k \in \Su$ representing the
incremental parallel transport.  The total holonomy is
\begin{equation}
  \label{eq:discrete-hol}
  \Hol_T(\gamma)
  \;=\;
  R_{N-1}\, R_{N-2}\, \cdots\, R_1\, R_0\,.
\end{equation}
The closure residual may be taken as the associated rotation angle
$\delta := \theta(\Hol_T(\gamma)) \in [0,2\pi]$, i.e.\ the magnitude of
the holonomy in the Lie algebra under the log map.
\end{definition}

\begin{definition}[Conjugate transport]
\label{def:conjugate-transport}
On $\Sthree \cong \Su$ every point $g$ admits a left action
$L_h : g \mapsto hg$ and a right action $R_h : g \mapsto gh$.  A
\emph{conjugate transport pair} $(T_L, T_R)$ consists of a left-acting
transport rule and a right-acting transport rule over the same
discretised path.  The two rules probe complementary aspects of the
manifold's curvature.  In a dual-lattice picture one may think of
$T_L$ and $T_R$ as defined on a lattice and its dual, respectively;
their combined holonomy deficit is invariant under the choice of
lattice refinement~\cite{Nakahara2003}.  More broadly, the conjugate
pairing ensures that the resulting closure residual is invariant under
refinement of the discretisation and choice of local frame.
\end{definition}

% ======================================================================
\section{Observer Compatibility $\neq$ Observer Dependence}
\label{sec:principle}
% ======================================================================

The central conceptual point of this note can be stated as a single
principle.

\begin{principle}[Compatibility without dependence]
\label{princ:compat}
Let $c$ be a closure invariant of a manifold~$M$.  Let $\mathcal{O}$
be a measurement process that is observer-compatible with~$c$
(\cref{def:observer-compat}).  Then:
\begin{enumerate}[nosep,label=(\alph*)]
  \item $c$ exists independently of whether $\mathcal{O}$ is
        instantiated;
  \item $\mathcal{O}$ does not alter the value of~$c$;
  \item $\mathcal{O}$'s compatibility constraints restrict the
        \emph{channel}, not the \emph{source}.
\end{enumerate}
\end{principle}

A schematic equation helps fix the distinction:
\begin{equation}
  \label{eq:compat-schema}
  \underbrace{c}_{\text{invariant}}
  \;\xleftarrow{\;\text{sampled by}\;}\;
  \underbrace{\mathcal{O}(f, \phi, n)}_{\text{observer process}}
  \qquad
  \text{where } f,\,\phi,\,n \text{ satisfy compatibility constraints.}
\end{equation}
Here $f$ is a sampling frequency, $\phi$ a phase-lock tolerance, and
$n$ a minimum cycle count.  The arrow points \emph{from} the observer
\emph{to} the invariant: the observer reads, it does not write.

\begin{remark}
This structure is analogous to the relationship between a mathematical
constant (say~$\pi$) and a numerical algorithm that computes its
digits.  The algorithm must satisfy convergence criteria (compatibility),
but $\pi$ does not depend on whether the algorithm runs.  Our claim is
that $\alphaFS$, understood as a closure invariant, stands in the same
relation to any physical process that measures it.
\end{remark}

% ======================================================================
\section{Phase-Compatible Sampling}
\label{sec:sampling}
% ======================================================================

We now make the compatibility constraints more concrete by framing
measurement as \emph{phase comparison over cycles}.

Suppose a physical oscillator of frequency~$f$ is coupled to a
channel in which the closure invariant~$\alphaFS$ is encoded as a
per-cycle phase offset~$\Delta\phi$:
\begin{equation}
  \label{eq:phase-offset}
  \Delta\phi \;=\; 2\pi\,\alphaFS \;\approx\; 4.6\times 10^{-2}\ \text{rad per cycle.}
\end{equation}
Here ``per cycle'' refers to one complete traversal of the
closure-defining transport loop, not a laboratory oscillation period.
After $n$ complete transport cycles the accumulated distinguishable
phase is
\begin{equation}
  \label{eq:accumulated}
  \Phi(n) \;=\; n\,\Delta\phi\,.
\end{equation}

For the measurement to \emph{resolve} $\alphaFS$ against a noise
background with phase jitter~$\sigma_\phi$ per cycle, we require
\begin{equation}
  \label{eq:snr}
  \frac{\Phi(n)}{\sqrt{n}\;\sigma_\phi}
  \;=\;
  \frac{\sqrt{n}\;\Delta\phi}{\sigma_\phi}
  \;\geq\;
  \Theta\,,
\end{equation}
where $\Theta$ is a detection threshold (e.g.\ $\Theta = 3$ for a
conventional $3\sigma$ criterion).  Rearranging:
\begin{equation}
  \label{eq:min-cycles}
  n
  \;\geq\;
  \left(\frac{\Theta\,\sigma_\phi}{\Delta\phi}\right)^{2}\!.
\end{equation}

Equation~\eqref{eq:min-cycles} is the \emph{cycle-count constraint}:
any observer-compatible process must sustain phase-locked oscillation
for at least $n$ cycles.  The \emph{frequency constraint} is
independent: $f$ must be high enough that $n$ cycles fit within the
observer's coherence window~$\tau_{\mathrm{coh}}$,
\begin{equation}
  \label{eq:freq-constraint}
  f
  \;\geq\;
  \frac{n}{\tau_{\mathrm{coh}}}\,.
\end{equation}

These two inequalities---\eqref{eq:min-cycles}
and~\eqref{eq:freq-constraint}---jointly define the phase-compatible
sampling regime.  Note that they constrain the observer, not the
invariant.

% ======================================================================
\section{Candidate Biological Window}
\label{sec:bio}
% ======================================================================

\begin{tcolorbox}[
  colback   = yellow!5,
  colframe  = orange!50!black,
  title     = {Status: Empirical Conjecture},
  fonttitle = \bfseries\small,
  boxrule   = 0.4pt,
  arc       = 2pt
]
\small
The content of this section is a \textbf{physiologically-facing
conjecture}.  It is \emph{not} a clinical claim and does not imply any
medical or therapeutic application.
\end{tcolorbox}

\medskip

Mammalian cortical networks sustain oscillatory activity in several
well-characterised frequency bands.  Of particular relevance is the
\emph{gamma band}, broadly spanning 30--100\,Hz, with a ``high-gamma''
sub-band in the 80--100\,Hz
range~\cite{Buzsaki2012, Fries2015}.  High-gamma oscillations are
associated with local cortical computation and short-range synchrony;
they are among the fastest sustained periodic signals observed in
neural tissue.

\begin{conjecture}[Candidate compatibility point]
\label{conj:87Hz}
If biological neural oscillators are modelled as phase-locked
sampling processes in the sense of \cref{sec:sampling}, then the
\emph{lowest frequency} at which the phase-compatible sampling
inequalities~\eqref{eq:min-cycles}--\eqref{eq:freq-constraint} can be
satisfied---given plausible neural phase-jitter and coherence-window
parameters---falls near $87$\,Hz, within the high-gamma band.
\end{conjecture}

The reasoning is as follows.  Empirical estimates of single-neuron
phase jitter in the gamma band place $\sigma_\phi$ in the range
$0.2$--$0.5$\,rad~\cite{Fries2015}.  Taking a moderate value
$\sigma_\phi \approx 0.3$\,rad and a $3\sigma$ threshold
($\Theta = 3$), equation~\eqref{eq:min-cycles} gives
\begin{equation}
  n \;\geq\;
  \left(\frac{3 \times 0.3}{2\pi\,\alphaFS}\right)^{2}
  \;\approx\;
  \left(\frac{0.9}{0.04586}\right)^{2}
  \;\approx\;
  385\;\text{cycles.}
\end{equation}
A cortical coherence window of $\tau_{\mathrm{coh}} \approx
3$--$5$\,s (sustained gamma bursts rarely exceed this duration)
yields, via~\eqref{eq:freq-constraint},
\begin{equation}
  f
  \;\geq\;
  \frac{385}{5\,\text{s}}
  \;=\;
  77\;\text{Hz}
  \qquad\text{(optimistic)},
  \qquad
  f
  \;\geq\;
  \frac{385}{3\,\text{s}}
  \;\approx\;
  128\;\text{Hz}
  \qquad\text{(conservative)}.
\end{equation}
The midpoint of this range sits near 87\,Hz, suggesting it as the
\emph{lowest stable compatibility point} under moderate assumptions.

\begin{remark}
This estimate is deliberately rough.  Its purpose is to show that the
compatibility framework produces a prediction in the right order of
magnitude and within a physiologically documented band---not to pin
down a precise frequency.  Refinement requires empirical work (see
\cref{sec:predictions}).  Nothing in this argument requires biological
systems to be unique or privileged observers, nor does it imply that
physical constants depend on the existence of life.
\end{remark}

% ======================================================================
\section{Predictions and Falsifiability}
\label{sec:predictions}
% ======================================================================

A compatibility claim is only useful if it can be tested.  We list
conditions that would support or refute the hypothesis of
\cref{conj:87Hz}.

\paragraph{Supporting evidence (would strengthen the claim).}
\begin{enumerate}[nosep,label=(S\arabic*)]
  \item High-gamma oscillations near 80--90\,Hz show measurably
        tighter phase-locking (lower $\sigma_\phi$) than neighbouring
        frequencies in the same cortical region.
  \item The number of phase-coherent cycles sustained at $\sim 87$\,Hz
        consistently exceeds the threshold~\eqref{eq:min-cycles}
        across subjects and species.
  \item Artificially extending the coherence window
        $\tau_{\mathrm{coh}}$ in controlled experimental paradigms
        (e.g.\ stimulus-locked task structure) shifts the inferred
        compatibility point downward, as predicted
        by~\eqref{eq:freq-constraint}.
\end{enumerate}

\paragraph{Refuting evidence (would weaken or falsify the claim).}
\begin{enumerate}[nosep,label=(R\arabic*)]
  \item No measurable phase-locking advantage exists for $\sim 87$\,Hz
        relative to other high-gamma frequencies.
  \item Cortical coherence windows are consistently too short
        ($\tau_{\mathrm{coh}} < 1$\,s) to satisfy the cycle-count
        constraint at any frequency below 200\,Hz.
  \item The per-cycle phase jitter $\sigma_\phi$ is so large that no
        biologically plausible frequency satisfies both constraints
        simultaneously.
\end{enumerate}

\paragraph{Suggested experimental directions.}
\begin{itemize}[nosep]
  \item \emph{EEG / MEG spectral analysis}: compare phase-locking
        value (PLV) across narrow sub-bands of high-gamma
        (70--100\,Hz) during sustained-attention tasks.
  \item \emph{Computational modelling}: simulate a recurrent cortical
        circuit with realistic noise and ask whether the sampling
        constraints~\eqref{eq:min-cycles}--\eqref{eq:freq-constraint}
        select a preferred oscillation frequency.
  \item \emph{Cross-species comparison}: test whether species with
        different gamma-band ranges show compatibility-point shifts
        consistent with their measured $\sigma_\phi$ and
        $\tau_{\mathrm{coh}}$.
\end{itemize}

% ======================================================================
\section{Relationship to Electromagnetism and the Standing-Wave Picture}
\label{sec:em}
% ======================================================================

The fine-structure constant governs the strength of the electromagnetic
interaction.  In the companion paper~\cite{Smart2025geometry} the
constant is derived as a property of $\Sthree$ geometry, with
electromagnetism entering only through the identification of U(1)
gauge phases with holonomy angles.  Here we briefly note how the
observer-compatibility framework connects to the electromagnetic
setting.

Consider a standing electromagnetic wave confined to a cavity whose
boundary conditions enforce a discrete spectrum.  Each mode of the
cavity is a physical oscillator with a well-defined frequency and
phase.  If the cavity dimensions are chosen so that the mode spacing
encodes the same angular deficit as the closure residual on~$\Sthree$,
then the cavity becomes an observer-compatible sampling device in the
sense of \cref{def:observer-compat}.

More precisely, let the cavity support a mode at frequency~$f_0$ such
that the round-trip phase shift equals $2\pi(1 - \alphaFS)$.  A
detector locked to~$f_0$ accumulates phase at a rate that is
commensurate with the holonomy of~$\gamma$; after $n$ cycles the
detector's integrated phase distinguishes $\alphaFS$ from zero in
exactly the manner described by~\eqref{eq:snr}.

This picture does not add new physics.  It restates the well-known
fact that precision measurements of~$\alphaFS$ (e.g.\ via the
anomalous magnetic moment of the electron~\cite{Aoyama2020}) rely on
phase-sensitive interferometric techniques.  What the compatibility
framework contributes is a \emph{structural} reading of why such
techniques work: they satisfy the phase-comparison constraints of
\cref{def:observer-compat}, and those constraints descend from the
geometry of the invariant itself.

The conjugate-transport picture (\cref{def:conjugate-transport}) maps
onto the dual descriptions of electromagnetic fields: the left- and
right-acting $\Su$ transports correspond, in a suggestive (but at
this stage heuristic) way, to the electric and magnetic components of
the field, related by Hodge duality.  Developing this correspondence
rigorously is beyond the scope of this note; we flag it as a direction
for future work.

% ======================================================================
\section{Conclusion}
\label{sec:conclusion}
% ======================================================================

This bridge note contributes three things to the programme initiated
in~\cite{Smart2025geometry}:

\begin{enumerate}[nosep]
  \item \textbf{A definition.}  Observer compatibility
        (\cref{def:observer-compat}) provides precise language for
        discussing how a measurement process can resolve a geometric
        invariant without making that invariant process-dependent.

  \item \textbf{A distinction.}  \Cref{princ:compat} separates
        compatibility (a property of the channel) from dependence
        (a property that would compromise the invariance of~$\alphaFS$).
        This distinction is essential for keeping the geometric
        derivation free of anthropic or teleological baggage.

  \item \textbf{A falsifiable conjecture.}  \Cref{conj:87Hz} uses
        the compatibility constraints to predict that $\sim 87$\,Hz
        sits at or near the lowest stable sampling point for
        biological oscillators.  The conjecture is testable with
        existing neuroscience instrumentation
        (\cref{sec:predictions}).
\end{enumerate}

What this note does \emph{not} do: it does not derive~$\alphaFS$, it
does not claim biological or conscious origin for physical constants,
and it does not make clinical or therapeutic assertions.  The invariant
is geometric; the compatibility condition is physical; the biological
application is conjectural.

% ── Bibliography ──────────────────────────────────────────────────────
\bibliographystyle{unsrt}
\bibliography{references}

\end{document}
